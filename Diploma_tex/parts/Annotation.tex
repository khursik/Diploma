\newpage
\setcounter{page}{2}

\begin{center}
    \section*{Аннотация}
\end{center}

%\begin{abstract}

    Анализ электроэнцефалографии (ЭЭГ) является важным инструментом в нейробиологии,
    нейронной инженерии (например, интерфейсы мозг-компьютер (ИМК)), а также имеет промышленное 
    применение. С развитием этих областей появляется всё большая потребность в анализе 
    ЭЭГ данных. Получение информации об электрической активности зон головного мозга
    и о том, каким образом эта активность влияет на классификацию типа решаемой человеком
    задачи является важным шагом на пути к тому, чтобы извлекать из ЭЭГ больше информации
    о функционировании мозга. Для достижения данной цели на имеющихся ЭЭГ данных мы
    рассмотрели каждый электрод в отдельности, чтобы ответить на следующие вопросы:
    (1) Как амплитуда биоэлектрической активности, регистрируемой каждым электродом
    с соответствующей ему зоны мозга, в отдельности влияет на результат классификации типа
    решаемой задачи? (2) Какие из них наибольшим образом влияют на результат?

    Используя методы машинного обучения и проанализировав полученные результаты, было
    выдвинуто предположение о нецелесообразности использования в дальнейшем полученных результатов,
    рассматривая электроды таким образом.

    % Оценив значения p-value натренированных классификаторов,
    % выдвинутую гипотезу о том, что,
    % рассматривая электроды отдельно друг от друга, можно получить результат, точность
    % которого будет свыше 60\%, следует отвергнуть.

    Оценив значения p-value натренированных классификаторов, было рассмотрено два подхода,
    результаты которых согласуются друг с другом, и установлено, что нету такого электрода,
    по которому бы получилось сделать предсказание с точностью свыше 60\%. Поэтому была
    исключена возможность определения активности человека по сигналу от одного электрода.
    

    % Анализ электроэнцефалографии (ЭЭГ) является важным инструментом в нейробиологии,
    % нейроинженерии (например, интерфейсы мозг - компьютер (ИМК)), а также имеет промышленное 
    % применение. С развитием этих областей появляется всё большая потребность в анализе 
    % ЭЭГ сигналов. Получение информации об электрической активности зон головного мозга
    % и о том, каким образом эта активность влияет на классификацию типа решаемой человеком
    % задачи является важным шагом на пути к тому, чтобы извлекать из ЭЭГ больше информации
    % о функционировании мозга. Для достижения данной цели на имеющихся ЭЭГ данных мы
    % рассмотрели каждый электрод в отдельности, чтобы ответить на следующие вопросы:
    % (1) Как амплитуда биоэлектрической активности, регистрируемой каждым электродом
    % с соответствующей ему зоны мозга, в отдельности влияет на результат классификации типа
    % решаемой задачи? (2) Какие из них наибольшим образом влияют на результат?
    % Затем была выдвинута следующая гипотеза: рассматривая электроды отдельно друг от друга,
    % можно получить результат, точность которого $> 60\%$.
    % \\В ходе данной работы, используя методы машинного обучения, была получена информация о том,
    % как каждый электрод в отдельности влияет на результат, и установлено, какие
    % из электродов наибольшим образом влияют на качество натренированного классификатора.
    % А также, получив среднее p-value по всем электродам $p_{val}\thicksim  0.064$, сделан вывод
    % о том, что выдвинутую гипотезу следует отвергнуть.

    % \begin{center}
    %     \textbf{Abstract}
    % \end{center}

    % Electroencephalography(EEG) analysis has been an important tool in neuro-science with 
    % applications in neuroscience, neural engineering (e.g. Brain-сomputer interfaces, BCI's),
    % and even commercial applications. With the development of these areas, there is an increasing
    % need for the analysis of EEG signals. Obtaining information about the electrical activity of
    % brain areas and how this activity affects the classification of the type of task being
    % solved by a person is an important step towards extracting more information about the
    % functioning of the brain from the EEG. To achieve this goal, based on the available EEG data,
    % we examined each electrode separately to answer the following questions: (1) How much does
    % the activity of each electrode individually affect the result? (2) Which electrodes have the
    % greatest influence on the prediction of the type of problem being solved? Then the following
    % hypothesis was put forward: considering the electrodes separately from each other, it is
    % possible to obtain a result with an accuracy more than 60\%.
    % Using machine learning methods, information was obtained on how each electrode individually
    % affects the result, and it was established which of the electrodes most affect the quality
    % of the trained classifier. And also, having obtained the average p-value for all electrodes
    % $p_{val}\thicksim 0.064$, it was concluded that the hypothesis put forward should be rejected.

%\end{abstract}