\newpage

\section{Заключение}

Получение информации об электрической активности зон головного мозга и о том,
каким образом эта активность влияет на классификацию типа решаемой человеком задачи
является важным шагом на пути к тому, чтобы извлекать из ЭЭГ больше информации о
функционировании мозга.

В данной работе были рассмотрены данные для каждого электрода в отдельности, чтобы посмотреть
на то, как активность каждой зоны мозга в отдельности влияет на предсказание типа решаемой
задачи. 

Проанализировав полученные результаты, мы рассмотрели два подхода, результаты которых
согласуются друг с другом, и пришли к выводу, что нету такого электрода, по которому бы
получилось сделать предсказание с точностью свыше 60\%. Поэтому мы исключили возможность
определить активность человека по сигналу от одного электрода.

Однако в связи с тем, что было протестировано достаточно большое количество электродов, то в
действительности полученная двумя подходами оценка является консервативной и для более точных
результатов можно было бы учесть «Look Elsewhere Effect» --- увеличение вероятности обнаружить хотя
бы один значимый сигнал при проверке большого числа независимых гипотез.

% Задавшись вопросом о релевантности полученных результатов, были
% проанализированы полученные результаты и показано, что следует отверг-
% нуть гипотезу о том, что можно получить качество предсказания натрени-
% рованного классификатора свыше 60\% при рассмотрении каждого электро-
% да в отдельности.

Полученные результаты говорят о нецелесообразности использования данных ЭЭГ при рассмотрении
электродов отдельно друг от друга, по крайней мере применяя линейные методы машинного
обучения, в частности логистическую регрессию.

