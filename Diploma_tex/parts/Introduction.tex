
\vspace*{3mm}
\section{Введение}

\vspace*{7mm}
Электроэнцефалография (ЭЭГ) --- метод исследования головного мозга, основанный на
регистрации его электрических потенциалов. ЭЭГ измеряет колебания напряжения в
результате ионного тока в нейронах головного мозга. Клинически электроэнцефалограмма
является графическим изображением спонтанной электрической активности мозга в течение
определенного периода времени, записанной с нескольких электродов мозга или
поверхности скальпа \cite{EEG}, (т.е. каждый электрод соответствует определённой
области мозга).

ЭЭГ широко используется в исследованиях, связанных с нейронной инженерией, неврологией
и биомедицинской инженерией (например, интерфейсы мозг-компьютер \cite{EEG_app1},
анализ сна \cite{EEG_app2}, обнаружение приступов эпилепсии \cite{EEG_app3}) из-за
относительно низкой финансовой стоимости. Классификация этих сигналов является важным шагом на пути к тому, чтобы сделать
использование ЭЭГ более практичным в применении и менее зависимым от подготовленных
специалистов. Типичный процесс подготовки классификации ЭЭГ включает в себя удаление 
глазодвигательных и мышечных артефактов, отбор признаков и классификацию.

На самом базовом уровне набор данных ЭЭГ состоит из объектов --- векторов действительных
значений, которые которые представляют генерируемые мозгом потенциалы, снятые с кожи
головы. Размерность каждого такого вектора, характеризуется числом электродов и
количеством диапазонов спектра каждого электрода.

На данный момент существует большое количество информации о применении традиционных
алгоритмов машинного обучения для распознавания типа решаемой человеком задачи
(например, \cite{emotion_class1}, \cite{emotion_class2}), в которых для предсказания
учитываются данные всех электродов в совокупности. В то же время, исследований, в которых
электроды рассматриваются по отдельности, не достаточно для того, чтобы располагать полной
информацией об электрической активности мозга. 

\newpage
\vspace*{10mm}
В данной работе электроды были рассмотрены отдельно друг от друга, чтобы ответить на следующие
вопросы: (1) Как амплитуда биоэлектрической активности, регистрируемой каждым электродом
с соответствующей ему зоны мозга, в отдельности влияет на результат классификации типа
решаемой задачи? (2) Какие из них наибольшим образом влияют на результат? (3) Можно
ли получить результат, точность которого свыше 60\%, рассматривая таким образом электроды?

Для получения ответов на вышеперечисленные вопросы были использованы стратегии для классификации
ЭЭГ с использованием линейных методов машинного обучения, а также методы предварительной обработки
данных ЭЭГ. Полученная информация может послужить отправной точкой для начального этапа проектирования
архитектуры в будущих приложениях машинного обучения для классификации ЭЭГ. 