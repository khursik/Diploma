\newpage

\section{Постановка задачи}

\subsection{Задача обучения классификатора}

Итак, у нас есть обучающая выборка, которой соответствует пара объект-ответ.
Объект описывается 63 вещественными признаками --- амплитудами колебаний
электромагнитной активности (см. главу \ref{sec:chapter_3_3}, выделено красным цветом),
а ответы --- это числа от 0 до 3, соответствующие типу решаемой задачи.


\begin{equation}
    \label{eq:eq_1}
    \textbf{Обучающая выборка:\:} X^l=(x_i, y_i)_{i=1}^l,\quad x_i\in \mathbb{R}^n,\quad y_i\in\lbrace 0, 1, 2, 3\rbrace,
\end{equation}
где $\quad l=10643,\quad n=63$.\\[5 mm]
Перед нами задача обучения с учителем, задача классификации.

\subsection{Цели и задачи}


В данной работе мы будем рассматривать электроды не все сразу как это уже исследовалось
во многих работах (например, \cite{emotion_class1},\cite{emotion_class2},\cite{EEG_app2},
\cite{EEG_app3}), а по отдельности.

Выдвигаем следующую гипотезу: 
рассматривая таким образом электроды, получим результат (качество предсказания
классификатора), точность которого свыше 60\% --- хотим получить ответы на следующие вопросы: (1) как
амплитуда биоэлектрической активности, регистрируемой электродом с конкретной зоны мозга,
в отдельности влияет на результат классификации типа решаемой задачи? (2) Какие из них
наибольшим образом влияют на результат? 

Проанализировав полученные результаты
сделаем выводы об истинности выдвинутой гипотезы, а, следовательно, и достоверности полученных результатов.